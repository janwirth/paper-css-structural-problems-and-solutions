\chapter{Introduction}
The results of a poll on \gls{css} Preprocessors with 13 000 responses conducted in 2012 by online magazine CSS-Tricks display a significant preference of those who tried preprocessors towards them.\footnote{cf. \cite{preprocessorpoll}}
The popularity of preprocessors can be considered as a symptom of insufficiency \gls{css} as a tool for modern web projects.
Artifacts of the search for effective and efficient methods to structure \gls{css} Code dates back to 2003.\footnote{cf. \cite{methodmails}}

In the recent years several different practices and sets of practices have been promoted by widely recognized companies and individuals of the web development community.
Among others, these include Yandex\footnote{cf. \cite{bem}}, Yahoo\footnote{cf. \cite{atomiccsssite}}.
In defiance of the abundance of methods, resources that provide comparative information based on thorough examination is scarce.
We will approach common problems in the development of \gls{css} caused by the features, quirks and shortcomings of the language.
In the context of these problems we will examine methods promoted as solutions.
The aim of this paper is to provide insight for frontend developers challenging best practices as well as a base for further research.

The focus lies on purely \gls{css} based solution.
This means that the usage of preprocessors and minification are not within the scope of this paper.
Also out of scope are:
\begin{itemize}
    \item {\normalfont \bfseries Rendering gerformance} due to probable insignificance in comparison to other issues discussed here.
    \item {\normalfont \bfseries Readability} due to a lack of measures for \gls{css}-specific readability. 
        Readability is not to be confused with complexity and issues arising from it which will be addressed in \autoref{sec:scopeleaks}.
\end{itemize}
% semantics: seo is not relevant
% readability



% out of scope
% readability
% state management
% utilities 

% terminology: imprecise
% practices may not apply to all

% CSS3-Standard

\chapter{Structural Problems of CSS}
In this chapter, we will identify structural problems of \gls{css} and explain negative effects with the help of code samples.
In that context we will examine the negative impact they have on both readability and scalability.
By readabilty we mean how hard it is to comprehend the source code, i.e. acquire and maintain knowledge of the system that enables a developer to modify it without unexpected side-effects.
A project is considered scalable here if increasing it's scale or complexity has little to no negative impacts on readability.
The problems covered here are selected based on problems the aforementioned promoted practices are trying to solve.
% validation of problem selection

\section{Scope Leaks}
CSS were created based on the design Principle \gls{soc} with the intent to separate content and visual style.
In contrast to inline styles and embedded stylesheets they are reusable across different \gls{html} documents.
The rules of a stylesheet referenced in an \gls{html}-document may apply to any part of the DOM if selectors match and the rule is not overwritten or shadowed.
Because the scope of \gls{css} rules is global, we can also describe them as `unscoped'.
When a stylesheet is viewed as a program that is executed in the context of the \gls{dom}, the individual \gls{css}-rules may be viewed as impure functions executed in given order, altering their arguments.\footnote{cf. \cite{linearabstractmachine} p.158}

Similar to a program with impure functions, the stylesheet with of unscoped styles can have unforeseen side effects.
This situation is also to consider when modifying such a stylesheet.
For example, when a change is made to a selector or the associated properties with the intention to modify the visual appearance of a specific element of the \gls{dom}, the rule created or modified can apply to other elements aswell.
This so called `scope leak'\footnote{cf. \cite{mpgcss}} can occur in any \gls{html}-Document that references this very stylesheet.
When the developer undertaking the changes is not aware of all applications of a specific rule, some of the visual changes are considered unforeseen.
We now illustrate the given situation, initially presenting a visualisation of a hypthetical \gls{dom}:

\begin{figure}[H]
  \centering
  \Tree[.body 
        [.form
          [.label E-Mail: ]
          [.input ]
          [.button
            [.span.label subscribe ]
          ]
        ]
        [....
          [.div.team-member
            [.img ]
            [.label.name ]
          ]
        ]
      ]
\end{figure}

The following rule was written with the intention to set the type on the `E-Mail'-label in bold weight.

\begin{verbatim}
label {
  font-weight: bold;
}
\end{verbatim}

Applying these rules achieves the intended effect, but also changes the weight of the type the name of a team member is set in.
The \verb ... signify an arbitrarly complex \gls{dom}-structure.
While it is obvious to us here that the rule will set the name of an employee in bold type, it may not be in an actual project.
Depending on the here omitted complexity and the tools at hand, of the \gls{dom}, the effort connected to locating the effects of a modification may be large.
The cascade and Specificity mechanism of \gls{css} adds some somplexity to possible side effects due to possible interactions between individual rules.
We assume that in any stylesheet that the document references occurs the following, more specific, unscoped rule:

\begin{verbatim}
.label{
  font-weight: light;
}
\end{verbatim}

Now the \verb label -rule does not take effect, because the \verb .label -rule has a higher priority.\footnote{cf. \cite{w3conselectors}}
To achieve the desired results, we need to undertake further modifications, e.g. increasing the specificity of the new rule, which may result in an increased complexity of the source code.

The risk of unforeseen side effects in the development of generally unscoped \gls{css} is increased with the complexity of the codebase and the number of both active and developers.
This is caused by potentially decreased readability and the increase in knowledge required to foresee side effects. 
Here we can draw a parallel between non-local rules and Wulf and Shaw's considerations of non-local variables as `a major contributing factor in programs which are difficult to understand' for similar reasons.\footnote{\cite{globalvariables} p.28}
% dynamic web-applications

However, global or `base' styles are not inherently considered harmful.
Their usage is suitable to provide representational consistency inside single aswell across multiple \gls{html} documents.\footnote{cf. \cite{mpgcss}}

\section{Violations of the \acrlong{dry} Principle}
The \gls{dry} Principle was formulated in 2000 by Hunt and Thomas.
It states that 
`Every piece of knowledge must have a single, unambiguous, authoritative representation within a system.'\footnote{\cite{pragmaticprogrammer} p. 27}
When we consider the following excerpt we can observe what is most likely a violation of \gls{dry}

\begin{verbatim}
button {
    color: red;
    padding: 5px;
}

a {
    color: red;
    text-decoration: none;
}
\end{verbatim}

Assuming that the intention of the developer was to give both links and buttons an identical color, this piece of knowledge has two authoritative representations here.
But it is possible that the red color may only be identical by coincidence, meaning that it different semantics within the context of either element.
This implies that both definitions will not necessarily be change at the same time.
However, if this is not the case, that repetition may cause additional cost during maintenance.
This is because, if the inherently singular piece of knowledge changes, the code has to be changed in more that one place.
Even automated tools like find and replace are prone to errors of their operators as the operation may require detailed knowledge of the target system.\footnote{cf. \cite{humanautomation} p.408}

There are multiple options to eliminating the problem that has just been examined.
Variables are, to this date, only fully supported by Firefox 31 and has been removed as an experimental feature in Google Chrome from version 33 to 34.\footnote{cf. \cite{cssvariables}}
To eliminate the duplication constants are sufficient.
Besides variables there are no native gls{css} means to emulate constants in their classical definition.

All common \gls{css}-Preprocessors provide a variable feature.
These variables are only variable during the actual processing and not at interpretation-time.
Thus preprocessor variables are effectively constants.\footnote{cf. \cite{wirthpreprocessors} p.27}
Using preprocessors allows eliminating repetition in the source code to some degree
but the repetition will still be observable in the processed \gls{css}.
Verou assumes that the usage of preprocessors may result in the developer `losing track of CSS filesize'.\footnote{cf. \cite{veroupreprocessors}}
This potential may not be as big when using preprocessed variables, but bigger for other language features like mixins.
% information size
% bandwidth for user

An approach to completely eliminate repetition in native \gls{css} code is Atomic \gls{css}, presentend in 2013 by Koblentz, frontend developer at Yahoo.
It advocates splitting the \gls{css} source code in the smallest possible parts.
This effectively decouples the \gls{css} from the structure of the \gls{html} document.
Atomic \gls{css} rules represent a specific visual appearance.
The class names of the individual rules are not semantic but representational.

% approach by frameworks, high reusability

\begin{verbatim}
.Mend-small {
    margin-left: 10px;
}
\end{verbatim}

In contrast to block-based approaches like \gls{suit} and \gls{bem} Atomic CSS is no scoping and no nesting.
Rules do not have a specific context and thus are highly reusable.\footnote{cf. \cite{atomiccssarticle}}
There are multiple tradeoffs to the Atomic CSS approach:

\begin{itemize}
    \item {\normalfont \bfseries Additional classes in markup:} Visual attributes are grouped through class assignment rather than below a specific selector.
        This means more classes need to be assigned to each element whose visual appearence deviates from default and base styles.
        The resulting markup may be considered bloated.

    \item {\normalfont \bfseries Non-semantic class names:} Because of the radical decoupling the \gls{css} rules are named by the visual attribute they yield. 
        The lack of scoping possibly impedes the developer's ability to localize the effect of changes.

    \item {\normalfont \bfseries Changing names and unused rules:} The W3C states that `Good names don't change'.\footnote{\cite{classsemantics}}
        The Usage of class names that inherit attribute values are a violation of \gls{dry}. 
        Replacing these attribute values with names like \verb small may eleminate the need to change class names when the actual attribute values change.
\end{itemize}

The development of the Atomic CSS approach has since been continued by Yahoo.
Current versions do not involve direct editing of \gls{css}-files.
The semantic hierarchy between class names and attributes is inversed.
Class names define the actual style attributes.
Atomizer, a build tool that extracts \gls{css} rules from the definitions in the markup is used to generate gls{css}-files.
Atomizer supports the usage of variables.
Whereas an overhead of unused rules in the \gls{css}-files is prevented by this approach, the markup is still studded with non-semantically named classes.\footnote{cf. \cite{atomiccsssite}}

% vs inline styles
% applying and de-applying classes, meaning of classes, logic has to be stored in module context
Hower
effectively a preprocessor
Atomic Design by ??? 

Modules / Components
modifiers on components or elements
balance between repeating 





no variables
repeated within the markup

respect to SOLID design principles\footnote{cf. \cite{solidcss}}
sub-classes = re-usability through dry
cascade

DRY Principle \footnote{cf. \cite{pragmaticprogrammer}}
low reusability
- specificity
- tight coupling with markup

DRY CSS: breaks code structure


% Vendor-prefixes

\section{Solutions}
After describing \gls{css}-specific structural problems, we will now introduce methods take from the promoted set of practices.
We note that the methods presented may not be suitable for any developer or project due to both personal preference or established conventions and practices.
The methods covered here are introduced and evaluated based on code samples.
In the evaluation we will also consider semanticity of selectors.
This refers to what a reader can infer about the contents of a rule by reading the associated selector alone.
We note that semantics here are not to be confused with microformats and semantic \gls{html}5-Elements and thus are not relevant to \gls{seo}.

\subsection{Scope Leaks}
To this date, while some browsers support scoped \verb <style> -tags, the CSS scoping module level 1 is still in draft.\footnote{cf. \cite{cssscopingmodule}}\footnote{cf. \cite{styletag}}
The \gls{w3c}-draft on CSS scoping defines a syntax that is similar to that of media queries.
However, media queries no not operate based on the \gls{dom}-structure but are scoping styles based on various properties of the rendering device.\footnote{cf. \cite{mediaqueries}}
Here it is possible to surround a set of rules with a selector that, if matched, limits these rules' scope to children of the matched element.
\begin{verbatim}
@scope div {
  span {
  color: blue;
  }
}
\end{verbatim}
In terms of limiting the visibility of rules this has the exact same effect as prefixing it with that selector.\footnote{cf. \cite{cssscopingmodule}}
% Atkins  
% syntactic sugar, other selector, tab atkins

A selector chain with more than one selector scopes the rule by the first selector:
\begin{verbatim}
.sidebar span {
  background: red;
}
.sidebar a {
  color: blue;
}
\end{verbatim}
\begin{figure}
\caption{The rules are scoped within the context of sidebar elements}
\end{figure}

The usage of this technique is often observed to limit the scope of styles to a specific page.
Coyier introduced and Escalante recommends decorating decorating the body tag of each \gls{html} document of a project with an ID.\footnote{cf. \cite{coyieridbody}}\footnote{cf. \cite{mpgcss}}
The major use case for page or page-type specific styles is the creation of visual exceptions among all pages.
Generally, page-specific styles may be used to define visual cues as visual orientation aids.
A concrete example for this is a visual design specification that distinguishes the header of an article page through a an opaque background instead of being  a transparent layer over a darker image.

\begin{verbatim}
header {
  background: transparent;
}
#article header {
  background: black;
}
\end{verbatim}

Coyier also exemplifies the use of page-specific styles through highlighting the currently selected entry of the main menu without changing the markup of the latter.

\begin{verbatim}

// index.html

<body id='about'>
  <nav>
    <ul>
      <li class='nav-item home'>
      <li class='nav-item about'>
    ...


// style.css

nav li {}

#about .nav-item.about {}

\end{verbatim}

However this technique should be applied with caution.
The markup and the styles in this example are coupled tightly and effectively violate the principle of \gls{soc}.
Each navigation entry requires an additional rule to function correctly in the context of the user interface.
Because this leads to bloat in \gls{css} code when the number of navigation entry is growing, a dynamic \gls{js} client- or server-side solution modifying the markup might be more when aiming for scalability.

The methodologies \gls{bem} and \gls{suit} share the concept of creating indivisble blocks in both .
Blocks encapsulate \gls{html} and \gls{css} as well as `other implementation technologies'\footnote{\cite{bem}}  and are also referred to as `components'.\footnote{cf. \cite{suit}}\footnote{cf. \cite{bem}} 
\gls{suit} is inspired by and may be viewed as being evolved from \gls{suit}.
Both have very similar conventions for the construction of \gls{css} rule selectors and the related markup.
The following example displays the construction of component and child-element selectors in \gls{suit}.
snake/dash/camel case

\begin{verbatim}
.BlockName {}

.BlockName-descendantName {}
\end{verbatim}
Whereas the delimiters can be chosen freely in \gls{bem} the naming conventions are more specific for \gls{suit}.
The reason for capitalising the root node selector is to avoid name collisions and thus expectedly better integration with existing code.\footnote{cf. \cite{bemvssuitquestion}}
The example shows that the scoping here does not utilise the cascade but works through prefixing the actual class names.
Descendent styles are applied to elements decorated with the classname \verb ComponentName-descendantName , regardless of their position in the \gls{dom}.

Escalante criticises this approach.
Instead of leveraging the functionality of the cascade, \gls{bem} and \gls{suit} annul it to `replicate it [...] in a less efficient manner'.\footnote{\cite{mpgcss}}
Inheriting the class name of the block root into the class name of a descendant causes repetitition and thus redundancy in the \gls{html}.\footnote{cf. \cite{mpgcss}}
{\slshape Title CSS} leverages the cascade to create a scope.
Through separating the block identifier as distinct class, repetition within the \gls{html}-markup is avoided:

\begin{verbatim}
.BlockName {}

.BlockName .descendantName{}
\end{verbatim}
Besides avoiding repetition, Cuthbert adds ease-of-writing as well as readability as rationales for {\slshape Title CSS} as naming convention.
After presenting methods for avoiding scope leaks in \gls{css}, methods to avoid redundancy within \gls{css} files are presented in the following section.

% performance
% utility
% - overriding 
% shadowing

\subsection{Violations of the DRY Principle}
There are several tools and methods available that allow eliminating repetitions in \gls{css}-code.
Variables are, to this date, only fully supported by Firefox 31 and has been removed as an experimental feature in Google Chrome from version 33 to 34.\footnote{cf. \cite{cssvariables}}
To eliminate the duplication constants are sufficient.
Besides variables there are no native gls{css} means to emulate constants in their classical definition.

All common \gls{css}-Preprocessors provide a variable feature.
These variables are only variable during the actual processing and not at interpretation-time.
Thus preprocessor variables are effectively constants.\footnote{cf. \cite{wirthpreprocessors} p.27}
Using preprocessors allows eliminating repetition in the source code to some degree
but the repetition will still be observable in the processed \gls{css}.
Verou assumes that the usage of preprocessors may result in the developer `losing track of CSS filesize'.\footnote{cf. \cite{veroupreprocessors}}
This potential may not be as big when using preprocessed variables, but bigger for other language features like mixins.
% information size
% bandwidth for user

An approach to completely eliminate repetition in native \gls{css} code is Atomic \gls{css}, presentend in 2013 by Koblentz, frontend developer at Yahoo.
It advocates splitting the \gls{css} source code in the smallest possible parts.
This effectively decouples the \gls{css} from the structure of the \gls{html} document.
Atomic \gls{css} rules represent a specific visual appearance.
The class names of the individual rules are not semantic but representational.

% approach by frameworks, high reusability

\lstset{language=CSS3,caption={Atomic CSS Rule}}
\begin{lstlisting}
.Mend-small {
    margin-left: 10px;
}
\end{lstlisting}

In contrast to block-based approaches like \gls{suit} and \gls{bem} Atomic CSS is no scoping and no nesting.
Rules do not have a specific context and thus are highly reusable.\footnote{cf. \cite{atomiccssarticle}}
There are multiple tradeoffs to the Atomic CSS approach:

\begin{itemize}
    \item {\normalfont \bfseries Additional classes in markup:} Visual attributes are grouped through class assignment rather than below a specific selector.
        This means more classes need to be assigned to each element whose visual appearence deviates from default and base styles.
        The resulting markup may be considered bloated.

    \item {\normalfont \bfseries Non-semantic class names:} Because of the radical decoupling the \gls{css} rules are named by the visual attribute they yield. 
        The lack of scoping possibly impedes the developer's ability to localize the effect of changes.

    \item {\normalfont \bfseries Changing names and unused rules:} The W3C states that `Good names don't change'.\footnote{\cite{classsemantics}}
        The Usage of class names that inherit attribute values are a violation of \gls{dry}. 
        Replacing these attribute values with names like \verb small may eleminate the need to change class names when the actual attribute values change.
\end{itemize}

The development of the Atomic CSS approach has since been continued by Yahoo.
Current versions do not involve direct editing of \gls{css}-files.
The semantic hierarchy between class names and attributes is inversed.
Class names define the actual style attributes.
Atomizer, a build tool that extracts \gls{css} rules from the definitions in the markup is used to generate gls{css}-files.
Atomizer supports the usage of variables.
Whereas an overhead of unused rules in the \gls{css}-files is prevented by this approach, the markup is still studded with non-semantically named classes.\footnote{cf. \cite{atomiccsssite}}



% vs inline styles
% applying and de-applying classes, meaning of classes, logic has to be stored in module context
Atomizer is an additional build tool which is why we do not examine this any further.


% Modules / Components
% modifiers on components or elements
% balance between repeating 





% no variables
% repeated within the markup
% 
% respect to SOLID design principles\footnote{cf. \cite{solidcss}}
% sub-classes = re-usability through dry
% cascade

% DRY Principle \footnote{cf. \cite{pragmaticprogrammer}}
% low reusability
% - specificity
% - tight coupling with markup
% 
% DRY CSS: breaks code structure


\chapter{Conclusion}
\gls{css} as a language is easy to learn and hard to master.
The speed at which tangible results can be produced when using \gls{css} is fairly high, but may decline quickly.
Because of that, the development process of \gls{css} should implement adequate and effective methods to prevent scope leaks and gratuitous repetition with the aim to improve readability and scalability.
Cutting the technical debt caused by structural problems of \gls{css} involves not only modifying stylesheets but also the \gls{html}-documents that reference them.
The cost of this may be increased through possible tight coupling of other software components with the \gls{html}-component.

The examination of the selected methods showed that there are effective solutions available for preventing scope leaks and confining gratuitious repetition.
However, the methods aimed at enforcing \gls{dry} are mutually exclusive, as both naming conventions and mechanics.
They exist on a scale that ranges from complete decoupling and maximum reusability to flexible modularity with limited reusability.
In addition the efficiency of Atomic \gls{css} is limited by page scoping methods.
The decision when to apply scoping methods and which naming or rather structure convention to select remains to be decided in concrete project context with respect to developer preferences and established conventions.
% alignment of methods

% cascade
