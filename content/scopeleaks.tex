\section{Scope Leaks}
CSS were created based on the design Principle \gls{soc} with the intent to separate content and visual style.
In contrast to inline styles and embedded stylesheets they are reusable across different \gls{html} documents.
The rules of a stylesheet referenced in an \gls{html}-document may apply to any part of the DOM if selectors match and the rule is not overwritten or shadowed.
Because the scope of \gls{css} rules is global, we can also describe them as `unscoped'.
When a stylesheet is viewed as a program that is executed in the context of the \gls{dom}, the individual \gls{css}-rules may be viewed as impure functions executed in given order, altering their arguments.\footnote{cf. \cite{linearabstractmachine} p.158}

Similar to a program with impure functions, the stylesheet with of unscoped styles can have unforeseen side effects.
This situation is also to consider when modifying such a stylesheet.
For example, when a change is made to a selector or the associated properties with the intention to modify the visual appearance of a specific element of the \gls{dom}, the rule created or modified can apply to other elements aswell.
This so called `scope leak'\footnote{cf. \cite{mpgcss}} can occur in any \gls{html}-Document that references this very stylesheet.
When the developer undertaking the changes is not aware of all applications of a specific rule, some of the visual changes are considered unforeseen.
We now illustrate the given situation, initially presenting a visualisation of a hypthetical \gls{dom}:

\begin{figure}[H]
  \centering
  \Tree[.body 
        [.form
          [.label E-Mail: ]
          [.input ]
          [.button
            [.span.label subscribe ]
          ]
        ]
        [....
          [.div.team-member
            [.img ]
            [.label.name ]
          ]
        ]
      ]
\end{figure}

The following rule was written with the intention to set the type on the `E-Mail'-label in bold weight.

\begin{verbatim}
label {
  font-weight: bold;
}
\end{verbatim}

Applying these rules achieves the intended effect, but also changes the weight of the type the name of a team member is set in.
The \verb ... signify an arbitrarly complex \gls{dom}-structure.
While it is obvious to us here that the rule will set the name of an employee in bold type, it may not be in an actual project.
Depending on the here omitted complexity and the tools at hand, of the \gls{dom}, the effort connected to locating the effects of a modification may be large.
The cascade and Specificity mechanism of \gls{css} adds some somplexity to possible side effects due to possible interactions between individual rules.
We assume that in any stylesheet that the document references occurs the following, more specific, unscoped rule:

\begin{verbatim}
.label{
  font-weight: light;
}
\end{verbatim}

Now the \verb label -rule does not take effect, because the \verb .label -rule has a higher priority.\footnote{cf. \cite{w3conselectors}}
To achieve the desired results, we need to undertake further modifications, e.g. increasing the specificity of the new rule, which may result in an increased complexity of the source code.

The risk of unforeseen side effects in the development of generally unscoped \gls{css} is increased with the complexity of the codebase and the number of both active and developers.
This is caused by potentially decreased readability and the increase in knowledge required to foresee side effects. 
Here we can draw a parallel between non-local rules and Wulf and Shaw's considerations of non-local variables as `a major contributing factor in programs which are difficult to understand' for similar reasons.\footnote{\cite{globalvariables} p.28}
% dynamic web-applications

However, global or `base' styles are not inherently considered harmful.
Their usage is suitable to provide representational consistency inside single aswell across multiple \gls{html} documents.\footnote{cf. \cite{mpgcss}}
