\section{Violations of the \acrlong{dry} Principle}
The \gls{dry} Principle was formulated in 2000 by Hunt and Thomas.
It states that 
`Every piece of knowledge must have a single, unambiguous, authoritative representation within a system.'\footnote{\cite{pragmaticprogrammer} p. 27}
When we consider the following excerpt we can observe what is most likely a violation of \gls{dry}

\begin{verbatim}
button {
    color: red;
    padding: 5px;
}

a {
    color: red;
    text-decoration: none;
}
\end{verbatim}

Assuming that the intention of the developer was to give both links and buttons an identical color, this piece of knowledge has two authoritative representations here.
But it is possible that the red color may only be identical by coincidence, meaning that it different semantics within the context of either element.
This implies that both definitions will not necessarily be change at the same time.
However, if this is not the case, that repetition may cause additional cost during maintenance.
This is because, if the inherently singular piece of knowledge changes, the code has to be changed in more that one place.
Even automated tools like find and replace are prone to errors of their operators as the operation may require detailed knowledge of the target system.\footnote{cf. \cite{humanautomation} p.408}
