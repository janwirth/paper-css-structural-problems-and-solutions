\chapter{Introduction}
\label{ch:intro}
The results of a poll on \gls{css} Preprocessors with 13 000 responses conducted in 2012 by online magazine CSS-Tricks display a significant preference of those who tried preprocessors towards them.\footnote{cf. \cite{preprocessorpoll}}
The popularity of preprocessors can be considered as a symptom of insufficiency \gls{css} as a tool for modern web projects.
Stylesheet size and structure do have little perceptible impact on both load time and rendering performance.\footnote{cf. \cite{soundersonselectorperformance}}\footnote{cf. \cite{atkinsondryscope}}
Browser inconsistencies have a relatively straightforward solution, namely normalize- or reset-stylesheets.\footnote{cf. \cite{meyeronreset}}\footnote{cf. \cite{gallagheronnormalize}}
Solutions to layout problems like collapsing margins or unexpected float behavior tend to be less straightforward, but are proven and undebatable.\footnote{cf. \cite{meyeronfloats}}\footnote{cf. \cite{meyeronmargins}}
However, artifacts of the search for effective and efficient methods to structure \gls{css} Code dates back to 2003.\footnote{cf. \cite{methodmails}}

In the recent years several different practices or sets of practices to structure stylesheets have been promoted by widely recognized companies and individuals of the web development community.
Among others, these include Yandex\footnote{cf. \cite{bem}}, Yahoo\footnote{cf. \cite{atomiccsssite}}.
In defiance of the abundance of methods and `best practices', resources that provide comparative information based on thorough examination are scarce.
We will approach common structural problems in the development of \gls{css} caused by the features, quirks and shortcomings of the language.
Consecutively, we will examine selected practices promoted as solutions for these problems.
We focus our analysis on CSS-only solutions but note relevant tools like preprocessors and template engines.
The aim of this paper is to provide comprehensible insight for frontend developers that want to challenge best practices.

The following conventions are used in this paper:
\verb ... signifies the presence of omitted code of arbitrary complexity.
File names like \verb \\ \verb index.html in code excerpts name the following lines of code.
Within an example the file \verb index.html references \verb style.css .
