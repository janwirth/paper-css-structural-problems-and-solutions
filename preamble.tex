% Hinweis: Optionen der Dokumentenklasse werden an alle folgenden \usepackage{package} Befehle weitergegeben
\documentclass[
	fontsize=12pt,
	paper=a4,
	parskip=half,
	twoside=false,
	numbers=noenddot,	% Kein Punkt am Ende einer Überschrift
	draft=false,			% Deckt Schwächen auf: overfull und full boxes werden markiert; Bilder werden nicht geladen
	bibliography=totoc,	% Literaturverzeichnis ins Inhaltsverzeichnis aufnehmen
	listof=totoc,		% Tabellen- und Abbildungsverzeichnis ins Inhaltsverzeichnis aufnehmen
	titlepage=true,		% Separate Titelseite; Gestaltung mit Hilfe der Titlepage-Umgebung
	% headsepline=true,	% Kopflinie aktivieren
	% footsepline=true,	% Fußlinie aktivieren
	abstracton			% Abstract aktivieren
]{scrreprt}

% Zeichenkodierung Ausgabe ist T1-Kodierung: Wichtig für die Ausgabe von Umlauten
\usepackage[T1]{fontenc}

% Schrift festlegen
\usepackage{fontspec}

\setmainfont[
	BoldFont=Arialb.ttf,
	ItalicFont=Ariali.ttf,
	BoldItalicFont=Arialbi.ttf
]{Arial.ttf}

\usepackage{titlesec}
\renewcommand{\sectfont}{\rmfamily}
% Sprachauswahl für Lokalisierungen und Silbentrennung
\usepackage[english]{babel}

% Zitate: Anführungszeichen automatisch anhand der Sprache wählen
\usepackage[babel=true]{csquotes}

% Source-Code-Listings
\usepackage{listings}
\input{../latex-listings-web/definitions/css3}
\usepackage{color}
\definecolor{lightgray}{rgb}{0.95, 0.95, 0.95}
\definecolor{darkgray}{rgb}{0.4, 0.4, 0.4}
\definecolor{purple}{rgb}{0.65, 0.12, 0.82}
\definecolor{editorGray}{rgb}{0.95, 0.95, 0.95}
\definecolor{editorOcher}{rgb}{1, 0.5, 0} % #FF7F00 -> rgb(239, 169, 0)
\definecolor{editorGreen}{rgb}{0, 0.5, 0} % #007C00 -> rgb(0, 124, 0)
\usepackage{upquote}
\lstset{%
  % General design
  backgroundcolor=\color{editorGray},
  basicstyle={\small\ttfamily},   
  frame=l,
  % line-numbers
  xleftmargin={0.75cm},
  numbers=left,
  stepnumber=1,
  firstnumber=1,
  numberfirstline=true,	
  % Code design
  identifierstyle=\color{black},
  keywordstyle=\color{blue}\bfseries,
  ndkeywordstyle=\color{editorGreen}\bfseries,
  stringstyle=\color{editorOcher}\ttfamily,
  commentstyle=\color{darkgray}\ttfamily,
  % Code
  language=CSS3,
  alsodigit={.:;},	
  tabsize=2,
  showtabs=false,
  showspaces=false,
  showstringspaces=false,
  extendedchars=true,
  breaklines=true,
  % German umlauts
  literate=%
  {Ö}{{\"O}}1
  {Ä}{{\"A}}1
  {Ü}{{\"U}}1
  {ß}{{\ss}}1
  {ü}{{\"u}}1
  {ä}{{\"a}}1
  {ö}{{\"o}}1
}


% BibTeX-Symbol
\usepackage{texnames}

%Farbpaket laden
\usepackage{xcolor}

% Symbole, z.B. Haken
\usepackage{pifont}

% Zeilen in Tabellen zusammenfassen
\usepackage{multirow}

% Spalten in Tabellen zusammenfassen
\usepackage{multicol}

% Silbentrennung kann bei bestimmten Wörten mit Hilfe von diesem Paket deaktiviert werden 
\usepackage{hyphenat}

% Abkürzungsverzeichnis
\usepackage[printonlyused, withpage]{acronym}

\usepackage{caption}
\DeclareCaptionFormat{custom}{
	\itshape#1#2#3
}
\DeclareCaptionFont{mdit}{\mdseries\itshape}
\captionsetup[table]{
	format=custom,
	labelsep=newline,
	justification=centering,
	font=small,
	labelfont=mdit,
	belowskip=12pt,
	aboveskip=18pt,
}
\usepackage{floatrow}
\floatsetup[table]{capposition=top}

% Tiefe des Inhaltsverzeichnisses
\setcounter{tocdepth}{2}

% Punkte im Inhaltsverzeichnis
\usepackage{tocstyle}
\usetocstyle{allwithdot}

% Zum Einbinden von PDF-Dateien.
\usepackage{pdfpages}

% Paket zum Anpassen von Kopf- und Fußzeilen
\usepackage[automark]{scrpage2}
% \setlength{\headheight}{1.1\baselineskip}
% Liniendicke
% \setheadsepline{0.1pt}
% \setfootsepline{0.1pt}

% Kopf- und Fusszeile löschen
\clearscrheadfoot
% Kopf- und Fusszeile aktivieren
\pagestyle{scrheadings}



% Fuss links
% \ifoot[\verfasser]{\verfasser}
% Fuss rechts
\ofoot[\pagemark]{\pagemark}

% Grafiken einbinden
\usepackage{graphicx}
% Pfad zu den Grafiken
\graphicspath{{resources/}}

% Seitenränder setzen
\usepackage[left=4cm, right=2cm, top=3cm, bottom=2cm, footskip=1cm]{geometry}
\renewcommand*\chapterheadstartvskip{\vspace*{0cm}}

% Zeilenabstand auf 1.5 setzen
\usepackage{setspace}
\onehalfspacing

% Literaturverzeichnis
\usepackage[backend=bibtex,style=verbose-ibid]{biblatex}
\bibliography{Literature.bib}
\renewcommand{\bibname}{Literaturverzeichnis}
%Referenz zu URLs
\usepackage{url}

% Glossar
\usepackage[acronym,toc,nonumberlist]{glossaries}
\makeglossaries
% Format Glossary
\usepackage{mfirstuc}
\renewcommand*{\glsgroupskip}{}
\renewglossarystyle{listdotted}{%
	\renewcommand*{\glossentry}[2]{
	\item[\glsentryitem{##1}] \glossentryname{##1} \dotfill \glossentrydesc{##1}
	}
}
\setglossarystyle{listdotted}
\makeglossaries

% Titel als Referenzierung verwenden
\usepackage{titleref}
\usepackage{cleveref}

% Währungen
\usepackage{textcomp}

% Fussnoten fortlaufend nummerieren.
\usepackage{chngcntr}
\counterwithout{footnote}{chapter}

% Persönliche Daten
\newcommand{\titel}{CSS - A survey of common problems and solutions}
\newcommand{\art}{Term paper of the third academic year}
\newcommand{\studienbereich}{Wirtschaft}
\newcommand{\studiengang}{Onlinemedien}
\newcommand{\verfasser}{Jan Wirth}
\newcommand{\kurs}{ON13}
\newcommand{\ausbildungsbetrieb}{visual4 GmbH}
\newcommand{\betreuer}{Prof. Dr. Arnulf Mester}
\newcommand{\abgabedatum}{}
\newcommand{\unterschrift}{\rule{5cm}{0.2pt}}

% Links- und PDF-Einstellungen
\usepackage[hidelinks]{hyperref}
\hypersetup{
	pdfauthor = {\verfasser},
	pdftitle = {\titel},
	pdfsubject = {\art},
	pdfkeywords = {},
	pdfstartview = {Fit},
	colorlinks = {false},
	breaklinks = {true},
	bookmarksopen = {true}
}

% Verhinderung von Schusterjunge und Hurenkind
\clubpenalty = 10000
\widowpenalty = 10000
\displaywidowpenalty = 10000

% Seitenzäler für große, römische Zahlen
\newcounter{RomanPagenumber}

% suppress hbox warnings
\hfuzz=5.002pt 

% Abkürzungen
\newcommand{\dash}{d.\,h. }
\newcommand{\zB}{z.\,B. }
\newcommand{\bzw}{\bzw. }
\newcommand{\uU}{u.U. }


% Zitate in neue Zeile rücken
\usepackage{breakcites}

% Symbole bzw. Dingbats
\usepackage{amssymb}
\usepackage{pifont}
\newcommand{\cmark}{\ding{51}}%
\newcommand{\xmark}{\ding{55}}%
\newcommand{ \rmark }{\ding{118}}%

\newcommand{\requiredMark}{\ding{108}}%
\newcommand{\optionalMark}{\ding{109}}%

% redefine textsc for arial
\renewcommand{\textsc}[1]{\textit{#1}}

% reduce hyphenation
\pretolerance=2000
\tolerance=3000 
\emergencystretch=10pt
\doublehyphendemerits=1000000

% multiple footnotes in a row
\usepackage[multiple]{footmisc}

% add table padding
\usepackage{array}
\setlength\extrarowheight{2pt}


\usepackage{tikz-qtree}
